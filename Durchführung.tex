\section{Durchführung}
Benutzt wurden die Programme \texttt{Gaussian} und \texttt{GaussianView}.
Zuerst wurde ein Brommethan-Molekül und ein Chlor Atom in den Reaktionsraum eingefügt. 
Das Chlor Atom wurde gegenüber des Brom-Atoms positioniert. 
Weiter wurde ein Maximaler Atomabstand von \qty{3}{\angstrom} gesetzt und die Reaktionskoordinate festgelegt.
Anschließend wurde der Scan der Totalen Energie gestartet.
Verwendet wurde der Datensatz \textit{STO-3G}, außerdem wurde eine allgemeine Ladung, des Systems, auf $-1$ eingestellt.
Das Resultat ist in \ref{abb:STO} und \ref{abb:cc} dargestellt. 
Das Minimum wurde anschließend als Anfangswert der Geometrieoptimierung verwendet.
So konnte das Wahreminimum ermittelt werden.
Weiter wurde der Punkt ermittelt, an dem der Übergangszustand liegt.
Dieses gesamte Verfahren wurde anschließend mit dem Größeren Datensatz \textit{cc-pVDZ} wiederhohlt.
