\section{Theoretischer Hintergrund}
Unser Versuch basiert auf den Grundlegenden erkenntnissen der Quantenmechanik.
Grundlegend ist die Annahme, das ein Komplett räumlich getrenntes System von Atomkernen und Elektronen eine Energieerniedrigung erfährt, wenn sich der Abstand verringert.
Grund ist hier die Anziehung zwischen den Unterschiedlichen Ladungen. 
Je kleiner diese Abstände werden, dester mehr müssen Effekte betrachtet werden die aus der Abstoßung zwischen Elektronen untereinander bzw. Atomkernen Untereinander entstehen.
Bei der Mathematischen Untersuchung der Energien die aus Anziehung und Abstoßung der Teilchen resultiert, wird Folgendes Problem schnell deutlich:
Die Energie eines Moleküls mit $N$ Atomen, ist somit abhängig von $3\cdot N-6$ (bzw. $3\cdot N-5$) Raumkoordinaten.
Einfache Visualisierungen sind hier nicht mehr Möglich.
Auch das Lösen dieser Gleichungen, wird so zunehmend Schwerer.
Zur Vereinfachung werden Schnitte entlang einer Reaktionskoordinate Aufgetragen.
Hier wird also die Energie des Systems, über den Abstand der Reaktionspartner dargestellt.
Dieses Diagramm, kann genutzt werden um Minima zu finden, diese entsprechen dann stabile Molekülstrukturen.
Die beschriebene Lösungsstrategie, wird auch als Geometrieoptimierung bezeichnet.
Nach einer Groben Einordnung, wo diese Minima in diesem Diagramm liegen, kann nun der Abstand so variiert werden, dass $\frac{dE}{dR}=0\wedge \frac{d^2E}{dR^2}>0$.
Weiter kann Mithilfe der Zweiten Ableitung eine Frequenzanalyse durchgeführt werden.
Hier wird die Eigenschwingung berechnet.
Werden nur Reele Werte für die Kraftkonstante gefunden, so ist das ein weiterer Beweis, dass an dieser Stelle ein Energieminimum vorliegt.
Möglich ist das dadurch, dass Nahe des Minimums, die Krümmung des Harmonischen und Anharmonischen Oszillator gleich ist.
Für die Suche nach einem Übergangszustand kann dieses Verfahren ebenfalls verwendet werden.
Der einzige Unterschied ist, dass die Zweite Ableitung hier kleiner als null sein muss und das die Eigenfrequenz einen imaginären Wet haben muss.
Damit eine Berechnung von einzelnen Energien möglich ist, müssen noch weitere folgende Näherungen Angenommen werden:
Zuerst die Born-oppenheimer Näherung.
Diese beruht auf der Erkenntniss, dass Atomkerne viel schwerer sind, als die Elektronen.
So kann angenommen werden, das die Elektronen sowieso instantan den Bewegungen des Kerns folgen und damit die Bewegung der Kerne vernachlässigt werden kann.
Weiter wird angenommen, dass die Wellenfunktion des Moleküls, das Produkt der Wellenfunktion der Orbitale ist.
Über die ununterscheidbarkeit von Elektronen kann nun auf eine Matrix geschlossen werden, welche sich aus dem Fock-Operator, den überlappungsintegralen und der Energie zusammensetzt.
Dies ist dann ein Eigenwert Problem, welches über systematische (Hartree-Fock Verfahren) Variation der MO-koeffizienten gelöst werden kann. 
